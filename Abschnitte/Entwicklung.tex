\section{\uppercase{Entwicklung}}
\subsection{Schnittstellen}
Hier werden die Schnittstellen beschrieben, die mithilfe vom REST-Service vom Server zur Verf\"ugung stehen. Diese basieren auf Standard-HTTP-Operationen.
\begin{enumerate}
\item Nutzeroperationen
\begin{itemize}
\item Erstellung eines Nutzers (Post)\\
Hier ist die Erstellung von einem Nutzer m\"oglich. Nur registrierte Nutzer k\"onnen die App benutzen. Daf\"ur braucht man die Parameter: 
\begin{itemize}
\item username 
\item password
\end{itemize}
Antwort 200: Die Operation war erfolgreich. Die Antwort enth\"alt die folgenden Werte:
\begin{itemize}
\item id (ID des home-Ordners)
\item home
	\begin{itemize}
	\item subfolders (id, name)
	\item files (id, name)
	\end{itemize}
\end{itemize}

Antwort 400: Schlechte Anfrage, falls der Nutzer ung\"ultige oder schon vorhandene Parameter eintr\"agt (Passwort zu kurz oder Nutzername zu kurz)

\item Nutzer im System einloggen (Post) \\
Hier ist das Einloggen von einem Nutzer im System m\"oglich. Diese Funktionnalit\"at ist nur m\"oglich, wenn der Nutzer schon registriert ist. Daf\"ur braucht man: 
\begin{itemize}
\item username 
\item password
\end{itemize}
Antwort 200: Die Operation war erfolgreich. Die Dateien, die zur\"uck geschickt sind, sind:
\begin{itemize}
\item token
\item id (ID des home-Ordners)
\item home
	\begin{itemize}
	\item subfolders (id, name)
	\item files (id, name)
	\end{itemize}
\end{itemize}
Antwort 400: Schlechte Anfrage, falls der Nutzer ung\"ultige Parameter eintr\"agt. (Nutzer nicht im System)
\item Delete (Meldet die aktuelle angemeldete Benutzersitzung ab)
Hier ist die Abmeldung von der Sitzung m\"oglich. Daf\"ur braucht man den Token vom Nutzer. \\
Antwort 200: Die Operation war erfolgreich. 
Antwort 400: Schlechte Anfrage, der angegebene Token ist nicht vergeben.
\end{itemize}

\item Dateioperationen
\begin{itemize}
\item Hochladen von Dateien (Post)
Hier ist das Hochladen von einer Datei m\"oglich. Daf\"ur braucht man: 
\begin{itemize}
\item token
\item folderId 
\item file
\end{itemize}
Antwort 200: Die Operation war erfolgreich. Die Antwort enth\"alt folgende Parameter:
\begin{itemize}
\item id (id des Ordners)
\item subfolders (id, name)
\item files (id, name)
\end{itemize}
Antwort 400: Schlechte Anfrage, Im Fall die Gr\"o{\ss}e der Datei gr\"o{\ss}er als 30Mb ist. \\
Antwort 403: Verboten (ung\"ultiger Token) \\
Antwort 404: Ordner nicht gefunden
\item Herunterladen von Dateien (Get)
Hier ist das Herunterladen von einer Datei m\"oglich. Daf\"ur braucht man:
\begin{itemize}
\item token
\item folderId (Id vom \"ubergeordneten Ordner)
\item fileId (Id von der Datei)
\end{itemize}
Antwort 200: Die Operation war erfolgreich. Die Antwort hat folgende Parameter:
\begin{itemize}
\item id (id des Ordners)
\item subfolders (id, name)
\item files (id, name)
\end{itemize}
Antwort 403: Verboten (ung\"ultiger Token) \\
Antwort 404: Datei / Ordner nicht gefunden
\item Bearbeitung vom Dateiname (Put)
Hier ist die Bearbeitung von einer Datei m\"oglich. Daf\"ur braucht man:
\begin{itemize}
\item token
\item folderId (Id vom \"ubergeordneten Ordner)
\item fileId (Id von der Datei)
\item name (Der Name von der Datei) 
\end{itemize}

Antwort 200: Die Operation war erfolgreich. Die Antwort hat folgende Parameter:
\begin{itemize}
	\item id (id des Ordners)
	\item subfolders (id, name)
	\item files (id, name)
\end{itemize}
Antwort 400: Schlechte Anfrage \\
Antwort 403: Verboten \\
Antwort 404: Nicht gefunden
\item L\"oschen von einer Datei (Delete)
Hier ist das L\"oschen von einer Datei m\"oglich. Daf\"ur braucht man: 
\begin{itemize}
\item token
\item folderId (Id vom \"ubergeordneten Ordner)
\item fileID (Id von der Datei)
\end{itemize}
Antwort 200: Die Operation war erfolgreich. 
Antwort 403: Verboten (ung\"ultige Token) \\
Antwort 404: Datei / Ordner nicht gefunden
\end{itemize}

\item Verzeichnisoperationen
\begin{itemize}
\item Herunterladen von einem Ordner (Get)
Hier ist das herunterladen von einem Ordner m\"oglich. Daf\"ur braucht man: 
\begin{itemize}
\item token
\item folderId (Id vom Ordner) 
\end{itemize}
Antwort 200: Die Operation war erfolgreich. Die Antwort hat folgende Parameter:
\begin{itemize}
	\item id (id des Ordners)
	\item subfolders (id, name)
	\item files (id, name)
\end{itemize}
Antwort 403: Verboten \\
Antwort 404: Nicht gefunden
\item Erstellen eines Ordners (Post) \\
Hier ist das Erstellen von einem Ordner m\"oglich. Daf\"ur braucht man:
\begin{itemize}
\item token
\item folderId (id des \"ubergeordneten Ordners)
\item folder (Name des neuen Ordners)
\end{itemize}
Antwort 200: Die Operation war erfolgreich. Die Antwort hat folgende Parameter:
\begin{itemize}
	\item id (id des Ordners)
	\item subfolders (id, name)
	\item files (id, name)
\end{itemize}
Antwort 400: Schlechte Anfrage, Im Fall die Gr\"o{\ss}e der Datei gr\"o{\ss}er als 30Mb ist. \\
Antwort 403: Verboten \\
Antwort 404: Nicht gefunden
\item Bearbeitung von einem Ordner (Put) \\
Hier ist die Bearbeitung von einem Ordner m\"oglich. Daf\"ur braucht man: 
\begin{itemize}
\item token
\item folderId (Id vom \"ubergeordneten Ordner) 
\item folder (ein neuer Name) 
\end{itemize}
Antwort 200: Die Operation war erfolgreich. Die Antwort hat folgende Parameter:
\begin{itemize}
	\item id (id des Ordners)
	\item subfolders (id, name)
	\item files (id, name)
\end{itemize}
Antwort 400: Schlechte Anfrage \\
Antwort 403: Verboten \\
Antwort 404: Nicht gefunden
\item Delete \\
Hier ist das L\"oschen von einem Ordner m\"oglich. Daf\"ur braucht man: 
\begin{itemize}
\item token
\item folderId (Id vom \"ubergeordneten Ordner) 
\end{itemize}
Antwort 200: Die Operation war erfolgreich.
Antwort 403: Verboten \\
Antwort 404: Nicht gefunden
\end{itemize}
\end{enumerate}

\subsection{klassische App}

Die klassische App wurde programmiert, um zu testen, wie man eine Android-App mit einem REST-Service verbindet. Daf\"ur wurden ein paar Bibliotheken ausprobiert. Die App setzt alle Funktionen um, die vorher festgelegt wurden. Diese Funktionen sind:
\begin{itemize}
	\item Datei-Upload
	\item Datei-Download
	\item Ordner erstellen
	\item Nutzer registrieren
	\item Nutzer anmelden
	\item Foto machen und hochladen
\end{itemize}

\noindent
