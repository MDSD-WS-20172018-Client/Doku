\section{\uppercase{Entwicklung}}
\subsection{Schnittstellen}
Hier werden die Schnittstellen beschrieben, die mithilfe vom Restservice vom Server zur verf\"ugung stehen.
\begin{enumerate}
\item Operationen auf einen Nutzer
	\begin{itemize}
	\item Post (erstellung von einem Nutzer) \\
Hier ist die Erstellung von einem Nutzer m\"oglich. Wenn ein Nutzer unsere App. benutzen will, muss er erstmal sich registrieren lassen. Daf\"ur braucht man: \\
	- Ein Username \\
	- Ein Passwort \\
Antwort 200: Die Operation war erfolgreich\\ 
Antwort 400: Schlechte Anfrage, falls der Nutzer ung\"ultige oder schon vorhandene Parameter eintr\"agt.  

	\item Post (Nutzer im System einloggen) \\
Hier ist das Einloggen von einem Nutzer im System m\"oglich. Diese Funktionnalit\"at ist nur m\"oglich, wenn der Nutzer schon registriert ist. Daf\"ur braucht man: \\
	- Ein Username \\
	- Ein Passwort \\
Antwort 200: Die Operation war erfolgreich \\ 
Antwort 400: Schlechte Anfrage, falls der Nutzer ung\"ultige oder schon vorhandene Parameter eintr\"agt.  
	\item Delete (Meldet die aktuelle angemeldete Benutzersitzung ab)
Hier ist die Abmeldung von der Sitzung m\"oglich. Daf\"ur braucht man den Token vom Nutzer \\
Antwort 200: Die Operation war erfolgreich \\ 
Antwort 400: Schlechte Anfrage, falls der Nutzer ung\"ultige oder schon vorhandene Parameter eintr\"agt.  
	\end{itemize}

\item Operationen auf eine Datei
	\begin{itemize}
	\item Post (Hochladen von Dateien)
Hier ist das Hochladen von einer Datei m\"oglich. Daf\"ur braucht man: \\
	- Der Token \\
	- Der FolderId \\
	- Die Datei \\
Antwort 200: Die Operation war erfolgreich \\ 
Antwort 400: Schlechte Anfrage, Im Fall die Gr\"o{\ss}e der Datei gr\"o{\ss}er als 30Mb ist. \\
Antwort 403: Verboten \\
Antwort 404: Nicht gefunden
	\item Get (Herunterladen von Dateien)
Hier ist das herunterladen von einer Datei m\"oglich. Daf\"ur braucht man: \\
	- Der Token \\
	- Der FolderId (Id vom \"ubergeordneten Ordner) \\
	- Die DateiId (Id von der Datei) \\
Antwort 200: Die Operation war erfolgreich \\ 
Antwort 403: Verboten \\
Antwort 404: Nicht gefunden
	\item Put (Bearbeitung vom Dateiname)
Hier ist die Bearbeitung von einer Datei m\"oglich. Daf\"ur braucht man: \\
	- Der Token \\
	- Der FolderId (Id vom \"ubergeordneten Ordner) \\
	- Die DateiId (Id von der Datei) \\
	- Der Name von der Datei \\
Antwort 200: Die Operation war erfolgreich \\ 
Antwort 400: Schlechte Anfrage \\
Antwort 403: Verboten \\
Antwort 404: Nicht gefunden
	\item Delete (Löschen von einer Datei)
Hier ist das Löschen von einer Datei m\"oglich. Daf\"ur braucht man: \\
	- Der Token \\
	- Der FolderId (Id vom \"ubergeordneten Ordner) \\
	- Die DateiId (Id von der Datei) \\
Antwort 200: Die Operation war erfolgreich \\ 
Antwort 403: Verboten \\
Antwort 404: Nicht gefunden
	\end{itemize}

\item Operationen auf ein verzeichnis
	\begin{itemize}
	\item Get
	\item Post
	\item Put
	\item Delete
	\end{itemize}
\end{enumerate}