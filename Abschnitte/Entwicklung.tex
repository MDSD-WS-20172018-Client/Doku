\section{\uppercase{Entwicklung}}
\subsection{Schnittstellen}
Hier werden die Schnittstellen beschrieben, die mithilfe vom Restservice vom Server zur Verf\"ugung stehen. Sie basieren auf Standard-HTTP-Operationen.
\begin{enumerate}
\item Nutzeroperationen
\begin{itemize}
\item Erstellung von einem Nutzer(Post) \\
Hier ist die Erstellung von einem Nutzer m\"oglich. Wenn ein Nutzer unsere App. benutzen will, muss er erstmal sich registrieren lassen. Daf\"ur braucht man: 
\begin{itemize}
\item Ein Username 
\item Ein Passwort 
\end{itemize}
Antwort 200: Die Operation war erfolgreich\\ 
Antwort 400: Schlechte Anfrage, falls der Nutzer ung\"ultige oder schon vorhandene Parameter eintr\"agt. 

\item Nutzer im System einloggen (Post) \\
Hier ist das Einloggen von einem Nutzer im System m\"oglich. Diese Funktionnalit\"at ist nur m\"oglich, wenn der Nutzer schon registriert ist. Daf\"ur braucht man: 
\begin{itemize}
\item Ein Username 
\item Ein Passwort 
\end{itemize}
Antwort 200: Die Operation war erfolgreich \\ 
Antwort 400: Schlechte Anfrage, falls der Nutzer ung\"ultige oder schon vorhandene Parameter eintr\"agt. 
\item Delete (Meldet die aktuelle angemeldete Benutzersitzung ab)
Hier ist die Abmeldung von der Sitzung m\"oglich. Daf\"ur braucht man den Token vom Nutzer \\
Antwort 200: Die Operation war erfolgreich \\ 
Antwort 400: Schlechte Anfrage, falls der Nutzer ung\"ultige oder schon vorhandene Parameter eintr\"agt. 
\end{itemize}

\item Dateioperationen
\begin{itemize}
\item Hochladen von Dateien(Post)
Hier ist das Hochladen von einer Datei m\"oglich. Daf\"ur braucht man: 
\begin{itemize}
\item Der Token
\item Der FolderId 
\item Die Datei
\end{itemize}
Antwort 200: Die Operation war erfolgreich \\ 
Antwort 400: Schlechte Anfrage, Im Fall die Gr\"o{\ss}e der Datei gr\"o{\ss}er als 30Mb ist. \\
Antwort 403: Verboten \\
Antwort 404: Nicht gefunden
\item Herunterladen von Dateien (Get)
Hier ist das herunterladen von einer Datei m\"oglich. Daf\"ur braucht man:
\begin{itemize}
\item Der Token
\item Der FolderId (Id vom \"ubergeordneten Ordner)
\item Datei (Id von der Datei)
\end{itemize}
Antwort 200: Die Operation war erfolgreich \\ 
Antwort 403: Verboten \\
Antwort 404: Nicht gefunden
\item Bearbeitung vom Dateiname (Put)
Hier ist die Bearbeitung von einer Datei m\"oglich. Daf\"ur braucht man:
\begin{itemize}
\item Der Token
\item Der FolderId (Id vom \"ubergeordneten Ordner)
\item Die Datei (Id von der Datei)
\item Der Name von der Datei 
\end{itemize}

Antwort 200: Die Operation war erfolgreich \\ 
Antwort 400: Schlechte Anfrage \\
Antwort 403: Verboten \\
Antwort 404: Nicht gefunden
\item Löschen von einer Datei (Delete)
Hier ist das L\"oschen von einer Datei m\"oglich. Daf\"ur braucht man: 
\begin{itemize}
\item Der Token
\item Der FolderId (Id vom \"ubergeordneten Ordner)
\item Die DateiID (Id von der Datei)
\end{itemize}
Antwort 200: Die Operation war erfolgreich \\ 
Antwort 403: Verboten \\
Antwort 404: Nicht gefunden
\end{itemize}

\item Verzeichnisoperationen
\begin{itemize}
\item Herunterladen von einem Ordner (Get)
Hier ist das herunterladen von einem Ordner m\"oglich. Daf\"ur braucht man: 
\begin{itemize}
\item Der Token
\item Der FolderId (Id vom Ordner) 
\end{itemize}
Antwort 200: Die Operation war erfolgreich \\ 
Antwort 403: Verboten \\
Antwort 404: Nicht gefunden
\item Hochladen von einem Ordner (Post) \\
Hier ist das Hochladen von einem Ordner m\"oglich. Daf\"ur braucht man:
\begin{itemize}
\item Der Token
\item Der FolderId 
\item Der Ordner 
\end{itemize}
Antwort 200: Die Operation war erfolgreich \\ 
Antwort 400: Schlechte Anfrage, Im Fall die Gr\"o{\ss}e der Datei gr\"o{\ss}er als 30Mb ist. \\
Antwort 403: Verboten \\
Antwort 404: Nicht gefunden
\item Bearbeitung von einem Ordner (Put) \\
Hier ist die Bearbeitung von einem Ordner m\"oglich. Daf\"ur braucht man: 
\begin{itemize}
\item Der Token
\item Der FolderId (Id vom \"ubergeordneten Ordner) 
\item Den Folder (z.B ein neur Name) 
\end{itemize}
Antwort 200: Die Operation war erfolgreich \\ 
Antwort 400: Schlechte Anfrage \\
Antwort 403: Verboten \\
Antwort 404: Nicht gefunden
\item Delete \\
Hier ist das L\"oschen von einem Ordner m\"oglich. Daf\"ur braucht man: 
\begin{itemize}
\item Der Token
\item Der FolderId (Id vom \"ubergeordneten Ordner) 
\end{itemize}
Antwort 200: Die Operation war erfolgreich \\ 
Antwort 403: Verboten \\
Antwort 404: Nicht gefunden
\end{itemize}
\end{enumerate}
