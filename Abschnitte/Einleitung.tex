\section{\uppercase{Einleitung}}
	\label{sec:introduction}
	
	\subsection{Motivation}
	Modellgetriebene Software-Entwicklung (MDSD) soll dazu dienen das Entwickeln von Anwendungen so zu vereinfachen, dass Menschen ohne Programmierkenntnisse, Anwendungen erstellen k\"onnen, die f\"ur einen speziellen Anwendungsbereich eingesetzt werden k\"onnen.\\
	Das Entwickeln einer Smartphone-App ist eine komplexe Aufgabe. Wenn diese App mit einem Server kommunizieren soll, dann wird die Komplexit\"at dieser Aufgabe noch erh\"oht.\\
	Um trotzdem Ergebnisse erzielen zu k\"onnen gibt es bereits Ans\"atze, mit deren Hilfe man durch Beschreibungen des Problems Code erzeugen kann, durch den eine lauff\"ahige Anwendung entsteht. 
	
	\subsection{Ziel}
	Das Ziel besteht darin eine Android-App mit Hilfe von Methoden der MDSD zu erstellen. Die App soll \"uber eine REST-Schnittstelle Dateien auf einen Server laden k\"onnen und Dateien von diesem Server runterladen k\"onnen. Damit verschiedene Nutzer den Dateitransfer-Dienst nutzen k\"onnen, soll auch eine Funktion bereitgestellt werden, die es erm\"oglicht, dass Nutzer sich registrieren, anmelden und abmelden k\"onnen. Diese Funktionen werden von dem Server bereitgestellt und sind auch \"uber eine REST-Schnittstelle nutzbar.\\
	Bei der Umsetzung dieses Vorhabens wird herausgestellt, was mit den zum Zeitpunkt der Arbeit verf\"ugbaren Mitteln m\"oglich ist und welche Grenzen es noch gibt.
	
	\subsection{Aufgabenstellung}
	Zum Vergleichen des Ergebnisses, das mit Methoden der MDSD erstellt wird, wird vorher ein Prototyp erstellt, der den vollen Funktionsumfang bietet und auf klassische Weise programmiert wird.\\
	Im ersten Schritt der Entwicklung werden Mittel zur modellgetriebenen Entwicklung eines Android-Klienten ausgesucht. Im zweiten Schritt wird versucht den Prototyp des Klienten zu entwickeln. Dabei wird der erzeugte Code anschlie\ss{}end in Android Studio ge\"offnet und kompiliert. Anschlie\ss{}end wird die App getestet und mit der klassisch programmierten App verglichen, um die aktuellen Grenzen aufzuzeigen und herauszufinden, ob der Prototyp die Anforderungen erf\"ullt.\\
	Um den Funktionsumfang erf\"ullen zu k\"onnen werden einige Teile bei der MDSD-App mit selbstgeschriebenen Code erg\"anzt.
	
	\subsection{Abgrenzung}
	In dieser Arbeit soll nur ein Android-Klient erstellt werden, der mit einem Server \"uber eine REST-Schnittstelle kommuniziert. Es ist nicht Bestandteil dieser Arbeit eine Server-Anwendung zu erstellen, die die REST-Schnittstelle zur Verf\"ugung stellt.\\
	Das Produkt dieser Arbeit ist ein Prototyp, der durch verschiedene MDSD-Methoden erstellt wird. Der Anteil des selber geschriebenen Codes wird dabei versucht m\"oglichst klein zu halten.
	
	\subsection{Ergebnis}