\section{\uppercase{Zusammenfassung und Ausblick}}
\noindent Das Werkzeug $\text{MD}^2$ hat sich in der verwendeten Version als nicht so m\"achtig erwiesen, wie es gedacht war, weshalb nur ein geringer Anteil der App mit $\text{MD}^2$ umgesetzt werden konnte. Die Entwicklung eines Feature Providers, der auf dem von $\text{MD}^2$ erzeugten Code weiterarbeitet bot eine M\"oglichkeit das Erzeugen von Quellcode selber umzusetzen, auch wenn der Feature Provider ein sehr eingeschr\"anktes Werkzeug ist, das nur bei diesem Projekt eingesetzt werden kann. Die Umsetzung aller gew\"unschten Funktionen konnte nicht f\"ur die App realisiert werden, die mit $\text{MD}^2$ erzeugt wurde. Allerdings gibt es eine klassisch programmierte App, die den vollen Funktionsumfang bietet, die als Grundlage f\"ur \"Uberlegungen zur Erweiterung des Feature Providers dienen kann. Zuk\"unftig k\"onnten noch weitere Funktionen umgesetzt werden, der Feature Provider k\"onnte allgemeiner gemacht werden, sodass er f\"ur beliebige Anwendungen anwendbar ist.\\
Methoden der MDSD k\"onnen gut eingesetzt werden, um Android Apps zu generieren. Daf\"ur haben Android Apps entsprechende Strukturen, die sich in der Aufteilung der Code-Elemente wiederfinden. Als Beispiel daf\"ur seinen die Lebenszyklusmethoden genannt und die Aufteilung der View und Activity in verschiedenen Dateien.