\documentclass[a4paper,twoside]{article}

\usepackage{epsfig}
\usepackage{subfigure}
\usepackage{calc}
\usepackage{amssymb}
\usepackage{amstext}
\usepackage{amsmath}
\usepackage{amsthm}
\usepackage{multicol}
\usepackage{pslatex}
\usepackage{apalike}
\usepackage{SciTePress}
\usepackage[small]{caption}

\subfigtopskip=0pt
\subfigcapskip=0pt
\subfigbottomskip=0pt

\begin{document}

\title{\uppercase{Entwicklung eines Dateitransfer-Klienten mittels Methoden der modellgetriebenen Software-entwicklung}}

\author{\authorname{Mario Kaulmann\sup{1}, Naoufel Frioui\sup{1} and Carole Noutchegueme\sup{1}}
\affiliation{\sup{1}Fachbereich Informatik und Medien, Technische Hochschule Brandenburg, Magdeburger Stra\ss{}e 50, Brandenburg an der Havel, Deutschland}
}

\keywords{MDSD, modellgetriebene Software-Entwicklung, md2, Dateitransfer, REST, Android}

\abstract{Das wird sp\"ater geschrieben}

\onecolumn \maketitle \normalsize \vfill

\section{\uppercase{Einleitung}}
\label{sec:introduction}

\subsection{Motivation}

\subsection{Ziel}
Das Ziel besteht darin eine Android-App mit Hilfe von Methoden der modellgetriebenen Software-Entwicklung zu erstellen. Die App soll \"uber eine REST-Schnittstelle Dateien auf einen Server
laden k\"onnen und Dateien von diesem Server runterladen k\"onnen. Damit verschiedene Nutzer den Dateitransfer-Dienst nutzen k\"onnen, soll auch eine Funktion bereitgestellt werden, die es erm\"oglicht, dass Nutzer sich registrieren, anmelden und abmelden k\"onnen. Diese Funktionen werden von dem Server bereitgestellt und sind auch \"uber eine REST-Schnittstelle nutzbar.\\
Bei der Umsetzung dieses Vorhabens wird herausgestellt, was mit den zum Zeitpunkt der Arbeit verf\"ugbaren Mittel m\"oglich ist und welche Grenzen es noch gibt.

\subsection{Aufgabenstellung}
Zum Vergleichen des Ergebnisses, das mit Methoden der modellgetriebenen Software-Entwicklung erstellt wird, wird vorher ein Prototyp erstellt, der den vollen Funktionsumfang bietet und auf klassische Weise programmiert wird.\\
Im ersten Schritt der Entwicklung werden Mittel zur modellgetriebenen Entwicklung eines Android-Klienten ausgesucht. Im zweiten Schritt wird versucht den Prototyp des Klienten zu entwickeln. Dabei wird der erzeugte Code anschließend in Android Studio ge\"offnet und kompiliert. Anschlie\ss{}end wird die App getestet und mit der klassisch programmierten App verglichen, um die aktuellen Grenzen aufzuzeigen und herauszufinden, ob der Prototyp die Anforderungen erf\"ullt.

\subsection{Abgrenzung}
In dieser Arbeit soll nur ein Android-Klient erstellt werden, der mit einem Server \"uber eine REST-Schnittstelle kommuniziert. Es ist nicht Bestandteil dieser Arbeit eine Server-Anwendung zu erstellen, die die REST-Schnittstelle zur Verf\"ugung stellt.\\

\subsection{Ergebnis}

\section{\uppercase{Werkzeuge zur Entwicklung}}

\subsection{$\text{MD}^2$}

\subsection{Swagger}

\section{\uppercase{Entwicklung}}

\section{\uppercase{Test}}

\section{\uppercase{Zusammenfassung}}

\vfill
\end{document}

